%
% Niniejszy plik stanowi przykład formatowania pracy magisterskiej na
% Wydziale MIM UW.  Szkielet użytych poleceń można wykorzystywać do
% woli, np. formatujac wlasna prace.
%
% Zawartosc merytoryczna stanowi oryginalnosiagniecie
% naukowosciowe Marcina Wolinskiego.  Wszelkie prawa zastrzeżone.
%
% Copyright (c) 2001 by Marcin Woliński <M.Wolinski@gust.org.pl>
% Poprawki spowodowane zmianami przepisów - Marcin Szczuka, 1.10.2004
% Poprawki spowodowane zmianami przepisow i ujednolicenie 
% - Seweryn Karłowicz, 05.05.2006
% Dodanie wielu autorów i tłumaczenia na angielski - Kuba Pochrybniak, 29.11.2016

% dodaj opcję [licencjacka] dla pracy licencjackiej
% dodaj opcję [en] dla wersji angielskiej (mogą być obie: [licencjacka,en])
\documentclass[licencjacka]{pracamgr}
\usepackage[polish]{babel}

% Dane magistranta:
\autor{Aleksander Mućk}{382184}


% Dane magistrantów:
%\autor{Aleksander Mućk}{342007}

\title{Algorytmy do klastrowania duplikacji genomowych}


\tytulang{Algorithms for the clustering of genomic duplication}

%kierunek: 
% - matematyka, informacyka, ...
% - Mathematics, Computer Science, ...
\kierunek{Bioinformatyka i Biologia Systemów}

% informatyka - nie okreslamy zakresu (opcja zakomentowana)
% matematyka - zakres moze pozostac nieokreslony,
% a jesli ma byc okreslony dla pracy mgr,
% to przyjmuje jedna z wartosci:
% {metod matematycznych w finansach}
% {metod matematycznych w ubezpieczeniach}
% {matematyki stosowanej}
% {nauczania matematyki}
% Dla pracy licencjackiej mamy natomiast
% mozliwosc wpisania takiej wartosci zakresu:
% {Jednoczesnych Studiow Ekonomiczno--Matematycznych}

% \zakres{Tu wpisac, jesli trzeba, jedna z opcji podanych wyzej}

% Praca wykonana pod kierunkiem:
% (podać tytuł/stopień imię i nazwisko opiekuna
% Instytut
% ew. Wydział ew. Uczelnia (jeżeli nie MIM UW))
\opiekun{dra hab. Pawła Góreckiego\\
  }

% miesiąc i~rok:
\date{Sierpień 2019}

%Podać dziedzinę wg klasyfikacji Socrates-Erasmus:
\dziedzina{ 
%11.0 Matematyka, Informatyka:\\ 
%11.1 Matematyka\\ 
%11.2 Statystyka\\ 
%11.3 Informatyka\\ 
%11.4 Sztuczna inteligencja\\ 
%11.5 Nauki aktuarialne\\
11.9 Inne nauki matematyczne i informatyczne
}

%Klasyfikacja tematyczna wedlug AMS (matematyka) lub ACM (informatyka)
\klasyfikacja{Computional biology, Applied computing, Life and medical sciences}

% Słowa kluczowe:
\keywords{duplikacja genu, drzewo genów, drzewo gatunków, analiza filogenetyczna, drzewo uzgadniające, Python, scenariusz ewolucyjny, strata genu, minimalizacja kosztu ewolucyjnego}

% Tu jest dobre miejsce na Twoje własne makra i~środowiska:
\newtheorem{defi}{Definicja}[section]

% koniec definicji

\begin{document}

\maketitle

%tu idzie streszczenie na strone poczatkowa

\begin{abstract}
   Niniejsza praca przedstawia propozycje rozwiązań algorytmicznych dla problemów klastrowania duplikacji genomowych w oparciu o scenariusze ewolucyjne. W części pierwszej wprowadzane są podstawowe pojęcia dotyczące drzew genów, gatunków, modeli ich uzgadniania oraz tworzenia scenariuszy ewolucyjnych. Omówiony został również problem przeliczania i klastrowania duplikacji genomowych. W części drugiej opisana została proponowana heurystyka wraz z przykładowymi testami oraz jej implementacją w języku Python.
\end{abstract}


\renewcommand{\contentsname}{Spis Treści}
\tableofcontents
%\listoffigures
%\listoftables

\chapter*{Wprowadzenie}
\addcontentsline{toc}{chapter}{Wprowadzenie}


Uzgadnianie drzew filogenetycznych jest, przez rozmiar danych i coraz bardziej skomplikowane modele, niezwykle złożone zarówno obliczeniowo jak i koncepcyjnie. Badania drzew genów i gatunków, a w szczególności zależności między nimi może odpowiedzieć na pytania w jaki sposób wyodrębniały się gatunki przez pryzmat zmian w ich genomie. Mimo wszystko jednak należy pamiętać, że pokrewieństwo gatunków nie zawsze implikuje pokrewieństwo genów, których drzewo ewolucyjne nie musi pokrywać się z drzewem zawierających je gatunków, które samo w sobie nie jest tak bardzo bardzo zróżnicowane jak drzewo genów. Tworzenie scenariuszy ewolucyjnych, dzięki którym możemy poznać w jaki sposób ewolucja genów wpływała na ewolucję gatunków jest zadaniem nietrywialnym. Potrzebne są narzędzia, które potrafiłyby ocenić scenariusze pod kątem ilości epizodów ewolucyjnych. Epizody, takie jak duplikacje genomowe, mogą być wyznacznikami prawdopodobieństwa danego scenariusza. Zagadnienie to jest jeszcze bardziej wymagające od uzgodnienia pojedynczego drzewa, ponieważ jeden scenariusz zawiera wiele drzew genów co wpływa na poziom złożoności obliczeń. W niniejszej pracy proponowany jest algorytm, który ocenia zbiór scenariuszy tworząc na ich podstawie jeden, którego koszt, liczony jako ilość duplikacji, będzie możliwie najmniejszy.
\\
Praca składa się z~czterech rozdziałów i~dodatków.
W~rozdziale \ref{r:pojecia} przedstawiono podstawowe pojęcia dotyczące drzew genów, drzew gatunków oraz modeli i scenariuszy ewolucyjnych.  
Rozdział~\ref{r:heurystyka} przedstawia propozycję heurystyki wraz~z jej testami na rzeczywistych danych.  W~rozdziale \ref{r:impl} opisano implementację i sposób użycia programu napisanego na podstawie przybliżonej we wcześniejszym rozdziale heurystyki.
Ostatni rozdział zawiera przemyślenia dotyczące możliwego użycia algorytmu i perspektyw jego rozwoju. W~dodatkach umieszczono fragmenty kodu, przykładowe dane wejściowe i~wyniki działania algorytmu.

\chapter{Podstawowe pojęcia}\label{r:pojecia}

W tym rozdziale poruszane są pojęcia i definicje niezbędne do zrozumienia problemistyki klastrowania duplikacji genomowych. 
\section{Wstęp biologiczny}


\section{Drzewa genów i gatunków}


Opis drzewa pojawiający się w danych przez Pana pracach.



Związki między genami przedstawia się za pomocą ukorzenionego, binarnego drzewa, nazywanego drzewem genów, którego liśćmi są geny opisane przynależnością do danego gatunku.
\begin{defi}\label{Drzewa genów}
  Definicja matematyczna.
\end{defi}

Związki między gatunkami przedstawia się za pomocą ukorzenionego, binarnego drzewa, nazywanego drzewem gatunków, którego węzły wewnętrzne są specjacjami, a liście to gatunki.
\begin{defi}\label{Drzewa gatunków}
  Definicja matematyczna.
\end{defi}

\section{Uzgodnienie drzewa}
Podstawowy opis idei.

\section{Modele scenariuszy ewolucyjnych}

Podstawowy opis idei. Podział ruchów:

\begin{defi}\label{TMOVE}
  TMOVE.
\end{defi}

\begin{defi}\label{CLOST}
  CLOST.
\end{defi}

Dozwolone modele:
\begin{enumerate}
\item PG
\item FHS;
\end{enumerate}

\section{Opis modeli}

\begin{figure}[tp]
  \centering
  \framebox{\vbox to 4cm{\vfil\hbox to
      7cm{\hfil\tiny.\hfil}\vfil}}
  \caption{Obrazek scenariusza}
\end{figure}

\chapter{Heurystyka}\label{r:heurystyka}

\section{Opis algorytmu}

Algorytm zakłada, że na wejściu dostępne są dane:
\begin{itemize}
\item Drzewa genów $G_1$ ... $G_k$,
\item $m_i$ scenariuszy drzewa $G_i$ opisanych jako wektory ${v_i}_1$, ${v_i}_2$ ... ${v_i}_m$ z wyliczonymi wartościami kosztu ewolucyjnego, mierzonego jako ilość duplikacji dla każdego węzła.
\end{itemize}
Istnieje wektor przybliżający rozwiązanie ME, który nazwijmy V* i który pasuje do każdego scenariusza. \\
Jednym z takich wektorów jest wektor $V_{max}$, który na każdej współrzędnej \textit{x} zawiera maksymalną wartość ${{v_i}_m}[x]$ dla każdego scenariusza \textit{m} w każdym drzewie genów $G_i$. Oczywiście nie jest to rozwiązanie najlepsze, ponieważ jest one kosztowne, ale pasuje do każdego scenariusza.

Bazując na wyliczonym wektorze $V_{max}$ heurystyka wylicza wektor V* i w pętli poprawia go w następujący sposób:
\begin{itemize}
\item Obniż jedna z wartości w wektorze V*,
\item Zaakceptuj w/w zmianę jeśli dla każdego drzewa genów istnieje scenariusz, który jest zgodny z takim wektorem epizodów,
\item Zakończ działanie jeśli nie da się poprawić żadnej współrzędnej.
\end{itemize}
Wybór współrzędnej może, zależnie od potrzeby, może być dokonywany w inny sposób:
\begin{itemize}
\item od końca wektora (od korzenia),
\item od początku (od liści),
\item losowo.
\end{itemize}

\section{Testy algorytmu na danych rzeczywistych}
\section{Testy algorytmu na danych symulowanych}
\chapter{Dokumentacja użytkowa i~opis implementacji}\label{r:impl}

Opis działania programu.

\chapter{Podsumowanie}

W~pracy przedstawiono pierwszy i dosyć intuicyjny pomysł na ocenę scenariuszy ewolucyjnych pod kątem ilości duplikacji. Należy jednak wspomnieć, że bazując samą idę da się znacząco usprawnić i zejść ze złożonością nawet do liniowej oceny, bez poprzedzającego kroku, w którym wyliczane są wszystkie scenariusze.  


\section{Perspektywy rozwoju}


Trudno przewidzieć wszystkie możliwości rozwoju, ale te bardziej
oczywiste można wskazać już teraz.  Są to:
\begin{itemize}
\item opisy naszych algorytmów optymalizacyjnych,
\item opisy naszych algorytmów optymalizacyjnych,
\item opisy naszych algorytmów optymalizacyjnych,
\end{itemize}

\section{Perspektywy wykorzystania}

DOPYTAĆ

\appendix

\chapter{Pętla programu zapisana w~języku Python wykonywana dla losowego wybierania indeksów}

\begin{verbatim}

		max_trees = []
        for scenario in self:
            all_dup_pref = [tree.duplication_prefix for tree n scenario]
            max_trees.append(self.rate_scenario(all_dup_pref))
        max_tree = self.rate_scenario(max_trees)

        if select_type == "random":

            index_list = [x for x in range(len(max_tree)) if x != 0]

            while index_list:
                index_list_position = random.randint(0, len(index_list) - 1)
                index = index_list[index_list_position]

                max_tree_temp = max_tree[:]
                max_tree_temp[index] -= 1

                for scenario in self:
                    for tree in scenario:
                        for i in range(len(tree.duplication_prefix)):
                            if max_tree_temp[i] - tree.duplication_prefix[i] < 0:
                                break
                        else:
                            break
                    else:
                        index_list.pop(index_list_position)
                        break
                else:
                    max_tree = max_tree_temp
            return max_tree, sum(max_tree)
\end{verbatim}

\chapter{Przykładowe drzewo gatunków}

\begin{center}
(prot,(fung,((chlo,embr),(arth,((acoe,anne),(echi,(chon,(oste,(amph,(moll,((mamm,(aves,rept)),agna)))))))))))
\end{center}

\begin{figure}[tp]
  \centering
  \framebox{\vbox to 4cm{\vfil\hbox to
      7cm{\hfil\tiny.\hfil}\vfil}}
  \caption{Wizualizacja drzewa gatunków}
\end{figure}

\chapter{Przykładowe drzewa genów}

{\obeylines %
(((amph,aves),mamm),chon)
((((acoe,mamm),chlo),fung),prot)
(((((echi,arth),mamm),embr),fung),prot)
}

\begin{figure}[tp]
  \centering
  \framebox{\vbox to 4cm{\vfil\hbox to
      7cm{\hfil\tiny.\hfil}\vfil}}
  \caption{Wizualizacja drzewa genów}
\end{figure}



\chapter{Przykładowy wynik działania programu
    (dla zbioru guigo)}

{\obeylines %
------------------FHS---------------------------
Data loaded. 0.0% of the data was corrupted: 0 of 53.
total random
([0, 0, 0, 0, 0, 0, 0, 0, 0, 0, 0, 0, 0, 0, 0, 0, 1, 0, 0, 0, 1, 1, 1, 1, 1, 1, 1, 1, 2, 3, 4], 18)
Done in 2.3286948204040527 .
start
([0, 0, 0, 0, 0, 0, 0, 0, 0, 0, 0, 0, 0, 0, 0, 0, 0, 0, 0, 0, 0, 0, 0, 0, 0, 0, 0, 0, 0, 0, 5], 5)
Done in 0.0676581859588623 .
end
([0, 0, 0, 0, 0, 0, 0, 0, 0, 0, 0, 0, 0, 0, 0, 0, 0, 0, 0, 0, 0, 0, 0, 0, 0, 0, 0, 1, 0, 0, 3], 4)
Done in 0.24111151695251465 .
index random
([0, 0, 0, 0, 0, 0, 0, 0, 0, 0, 0, 0, 0, 0, 0, 0, 0, 0, 0, 0, 0, 0, 0, 0, 0, 0, 0, 0, 0, 2, 3], 5)
Done in 0.0971217155456543 .
------------------PG----------------------------
Data loaded. 0.0% of the data was corrupted: 0 of 53.
total random
([0, 0, 0, 0, 0, 0, 0, 0, 0, 0, 0, 0, 0, 0, 0, 0, 0, 0, 0, 0, 0, 0, 1, 1, 0, 0, 0, 1, 1, 1, 3], 8)
Done in 0.18230891227722168 .
start
([0, 0, 0, 0, 0, 0, 0, 0, 0, 0, 0, 0, 0, 0, 0, 0, 0, 0, 0, 0, 0, 0, 1, 0, 0, 0, 0, 1, 1, 0, 3], 6)
Done in 0.006891012191772461 .
end
([0, 0, 0, 0, 0, 0, 0, 0, 0, 0, 0, 0, 0, 0, 0, 0, 0, 0, 0, 0, 0, 0, 1, 0, 0, 0, 0, 1, 1, 0, 3], 6)
Done in 0.006652355194091797 .
index random
([0, 0, 0, 0, 0, 0, 0, 0, 0, 0, 0, 0, 0, 0, 0, 0, 0, 0, 0, 0, 0, 0, 1, 0, 0, 0, 0, 1, 1, 0, 3], 6)
Done in 0.007452487945556641 .
}

\begin{thebibliography}{99}
\addcontentsline{toc}{chapter}{Bibliografia}


\bibitem[Zen69]{heu} Zenon Zenon, \textit{Użyteczne heurystyki
    w~analizie}, Młody Technik, nr~11, 1969.

\end{thebibliography}

\end{document}


%%% Local Variables:
%%% mode: latex
%%% TeX-master: t
%%% coding: latin-2
%%% End:
