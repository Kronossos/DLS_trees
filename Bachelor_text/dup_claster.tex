%
% Niniejszy plik stanowi przykład formatowania pracy magisterskiej na
% Wydziale MIM UW.  Szkielet użytych poleceń można wykorzystywać do
% woli, np. formatujac wlasna prace.
%
% Zawartosc merytoryczna stanowi oryginalnosiagniecie
% naukowosciowe Marcina Wolinskiego.  Wszelkie prawa zastrzeżone.
%
% Copyright (c) 2001 by Marcin Woliński <M.Wolinski@gust.org.pl>
% Poprawki spowodowane zmianami przepisów - Marcin Szczuka, 1.10.2004
% Poprawki spowodowane zmianami przepisow i ujednolicenie 
% - Seweryn Karłowicz, 05.05.2006
% Dodanie wielu autorów i tłumaczenia na angielski - Kuba Pochrybniak, 29.11.2016

% dodaj opcję [licencjacka] dla pracy licencjackiej
% dodaj opcję [en] dla wersji angielskiej (mogą być obie: [licencjacka,en])
\documentclass[licencjacka]{pracamgr}
\usepackage[polish]{babel}

% Dane magistranta:
\autor{Aleksander Mućk}{382184}


% Dane magistrantów:
%\autor{Aleksander Mućk}{342007}

\title{Algorytmy do klastrowania duplikacji genomowych}


\tytulang{Algorithms for the clustering of genomic duplication}

%kierunek: 
% - matematyka, informacyka, ...
% - Mathematics, Computer Science, ...
\kierunek{Bioinformatyka i Biologia Systemów}

% informatyka - nie okreslamy zakresu (opcja zakomentowana)
% matematyka - zakres moze pozostac nieokreslony,
% a jesli ma byc okreslony dla pracy mgr,
% to przyjmuje jedna z wartosci:
% {metod matematycznych w finansach}
% {metod matematycznych w ubezpieczeniach}
% {matematyki stosowanej}
% {nauczania matematyki}
% Dla pracy licencjackiej mamy natomiast
% mozliwosc wpisania takiej wartosci zakresu:
% {Jednoczesnych Studiow Ekonomiczno--Matematycznych}

% \zakres{Tu wpisac, jesli trzeba, jedna z opcji podanych wyzej}

% Praca wykonana pod kierunkiem:
% (podać tytuł/stopień imię i nazwisko opiekuna
% Instytut
% ew. Wydział ew. Uczelnia (jeżeli nie MIM UW))
\opiekun{dra hab. Pawła Góreckiego\\
  }

% miesiąc i~rok:
\date{Sierpień 2019}

%Podać dziedzinę wg klasyfikacji Socrates-Erasmus:
\dziedzina{ 
%11.0 Matematyka, Informatyka:\\ 
%11.1 Matematyka\\ 
%11.2 Statystyka\\ 
%11.3 Informatyka\\ 
%11.4 Sztuczna inteligencja\\ 
%11.5 Nauki aktuarialne\\
11.9 Inne nauki matematyczne i informatyczne
}

%Klasyfikacja tematyczna wedlug AMS (matematyka) lub ACM (informatyka)
\klasyfikacja{Computional biology, Applied computing, Life and medical sciences}

% Słowa kluczowe:
\keywords{duplikacja genu, drzewo genów, drzewo gatunków, analiza filogenetyczna, drzewo uzgadniające, Python, scenariusz ewolucyjny, strata genu, minimalizacja kosztu ewolucyjnego}

% Tu jest dobre miejsce na Twoje własne makra i~środowiska:
\newtheorem{defi}{Definicja}[section]

% koniec definicji

\begin{document}

\maketitle

%tu idzie streszczenie na strone poczatkowa

\begin{abstract}
   Niniejsza praca przedstawia podstawowe propozycje rozwiązań algorytmicznych dla problemów klastrowania duplikacji genomowych w oparciu o scenariusze ewolucyjne. W części pierwszej wprowadzane są pojęcia dotyczące drzew genów, gatunków, modeli ich uzgadniania oraz tworzenia scenariuszy ewolucyjnych. Omówiony został również problem przeliczania i klastrowania duplikacji genomowych w scenariuszach ewolucyjnych. W części drugiej opisana została proponowana heurystyka wraz z przykładowymi testami oraz jej implementacja w języku Python.
\end{abstract}


\renewcommand{\contentsname}{Spis Treści}
\tableofcontents
%\listoffigures
%\listoftables

\chapter*{Wprowadzenie}
\addcontentsline{toc}{chapter}{Wprowadzenie}


Uzgadnianie drzew filogenetycznych jest, przez rozmiar danych i coraz bardziej skomplikowane modele, niezwykle złożone zarówno obliczeniowo jak i koncepcyjnie. Badania drzew genów i gatunków, a w szczególności zależności między nimi może odpowiedzieć na pytania w jaki sposób kształtuje się życie. Mimo wszystko jednak należy pamiętać, że pokrewieństwo gatunków nie zawsze implikuje pokrewieństwo genów, których drzewo ewolucyjne nie musi pokrywać się z drzewem zawierających je gatunków. Tworzenie scenariuszy ewolucyjnych, dzięki którym możemy poznać w jaki sposób ewolucja genów wpływała na ewolucję gatunków jest zadaniem nietrywialnym. Potrzebne są narzędzia, które będą potrafiłyby ocenić scenariusze są


Krótki opis zależności na bardzo podstawowym poziomie.


Praca składa się z~czterech rozdziałów i~dodatków.
W~rozdziale \ref{r:pojecia} przedstawiono podstawowe pojęcia dotyczące drzew genów, drzew gatunków oraz modeli i scenariuszy ewolucyjnych.  
Rozdział~\ref{r:heurystyka} przedstawia propozycję heurystyki wraz~z jej testami na rzeczywistych danych.  W~rozdziale \ref{r:impl} opisano implementację i sposób użycia programu napisanego na podstawie przybliżonej we wcześniejszym rozdziale heurystyki.
Ostatni rozdział zawiera przemyślenia dotyczące możliwego użycia algorytmu i perspektyw jego rozwoju. W~dodatkach umieszczono fragmenty kodu, przykładowe dane wejściowe i~wyniki działania algorytmu.

\chapter{Podstawowe pojęcia}\label{r:pojecia}

Pojęciem pierwotnym dla scenatiuszy ewolucy

\section{Drzewa genów i gatunków}
Opis drzewa ogólny



\begin{defi}\label{Drzewa genów}
  Opis.
\end{defi}

\begin{defi}\label{Drzewa gatunków}
  Opis.
\end{defi}

\section{Uzgodnienie drzewa}
Podstawowy opis idei.

\section{Modele scenariuszy drzew}

Podstawowy opis idei. Podział ruchów:

\begin{defi}\label{TMOVE}
  Opis.
\end{defi}

\begin{defi}\label{CLOST}
  Opis.
\end{defi}

Dozwolone modele:
\begin{enumerate}
\item PG
\item FHS;
\end{enumerate}

\begin{figure}[tp]
  \centering
  \framebox{\vbox to 4cm{\vfil\hbox to
      7cm{\hfil\tiny.\hfil}\vfil}}
  \caption{Obrazek scenariusza}
\end{figure}

\chapter{Heurystyka}\label{r:heurystyka}

Dane na wejściu : 
drzewa genów $G_1$...$G_k$
dla każdego Gi jest mi scenariuszy opisanych jako wektory vi1 vi2 ... vim z epizodami (wg outputu ze skryptu)

Zróbmy wektor przybliżające rozwiązanie ME, nazwijmy je v*, które pasuje do każdego scenariusza. Wystarczy brać max po współrzędnych z każdego wektora v....
Oczywiście jest to rozwiązanie słabe bo kosztowne ale pasuje do każdego scenariusza.

Teraz w pętli spróbujmy poprawiać współrzędne z wektora v*:
- obniż jedna z wartości w wektorze v
- zaakceptuj w/w zmianę jeśli dla każdego drzewa genów istnieje scenariusz, który jest zgodny z takim wektorem epizodów
Zakończ jeśli nie da się poprawić żadnej współrzędnej.

Wybór współrzędnej do obniżki: testować np. od końca wektora (od korzenia), od początku (od liści), losowo.

\chapter{Dokumentacja użytkowa i~opis implementacji}\label{r:impl}

Opis działania programu.

\chapter{Podsumowanie}

W~pracy przedstawiono pierwszą heurystykę.

Trudno przewidzieć wszystkie nowe możliwości, ale te co bardziej
oczywiste można wskazać już teraz.  Są to:
\begin{itemize}
\item opisy naszych algorytmów optymalizacyjnych,
\item opisy naszych algorytmów optymalizacyjnych,
\item opisy naszych algorytmów optymalizacyjnych,
\end{itemize}

\section{Perspektywy wykorzystania}

DOPYTAĆ

\appendix

\chapter{Główna pętla programu zapisana w~języku Python}

\begin{verbatim}

		max_trees = []
        for scenario in self:
            all_dup_pref = [tree.duplication_prefix for tree n scenario]
            max_trees.append(self.rate_scenario(all_dup_pref))
        max_tree = self.rate_scenario(max_trees)

        if select_type == "random":

            index_list = [x for x in range(len(max_tree)) if x != 0]

            while index_list:
                index_list_position = random.randint(0, len(index_list) - 1)
                index = index_list[index_list_position]

                max_tree_temp = max_tree[:]
                max_tree_temp[index] -= 1

                for scenario in self:
                    for tree in scenario:
                        for i in range(len(tree.duplication_prefix)):
                            if max_tree_temp[i] - tree.duplication_prefix[i] < 0:
                                break
                        else:
                            break
                    else:
                        index_list.pop(index_list_position)
                        break
                else:
                    max_tree = max_tree_temp
            return max_tree, sum(max_tree)
\end{verbatim}

\chapter{Przykładowe drzewo gatunków}

\begin{center}
(prot,(fung,((chlo,embr),(arth,((acoe,anne),(echi,(chon,(oste,(amph,(moll,((mamm,(aves,rept)),agna)))))))))))
\end{center}

\chapter{Przykładowe drzewa genów}

{\obeylines %

(((aves,mamm),amph),chon)
(((amph,aves),mamm),chon)
((((acoe,mamm),chlo),fung),prot)
(((((echi,arth),mamm),embr),fung),prot)
((((echi,arth),mamm),prot),embr)
}



\chapter{Przykładowy wynik działania programu
    (dla zbioru guigo)}

{\obeylines %
------------------FHS---------------------------
Data loaded. 0.0% of the data was corrupted: 0 of 53.
total random
([0, 0, 0, 0, 0, 0, 0, 0, 0, 0, 0, 0, 0, 0, 0, 0, 1, 0, 0, 0, 1, 1, 1, 1, 1, 1, 1, 1, 2, 3, 4], 18)
Done in 2.3286948204040527 .
start
([0, 0, 0, 0, 0, 0, 0, 0, 0, 0, 0, 0, 0, 0, 0, 0, 0, 0, 0, 0, 0, 0, 0, 0, 0, 0, 0, 0, 0, 0, 5], 5)
Done in 0.0676581859588623 .
end
([0, 0, 0, 0, 0, 0, 0, 0, 0, 0, 0, 0, 0, 0, 0, 0, 0, 0, 0, 0, 0, 0, 0, 0, 0, 0, 0, 1, 0, 0, 3], 4)
Done in 0.24111151695251465 .
index random
([0, 0, 0, 0, 0, 0, 0, 0, 0, 0, 0, 0, 0, 0, 0, 0, 0, 0, 0, 0, 0, 0, 0, 0, 0, 0, 0, 0, 0, 2, 3], 5)
Done in 0.0971217155456543 .
------------------PG----------------------------
Data loaded. 0.0% of the data was corrupted: 0 of 53.
total random
([0, 0, 0, 0, 0, 0, 0, 0, 0, 0, 0, 0, 0, 0, 0, 0, 0, 0, 0, 0, 0, 0, 1, 1, 0, 0, 0, 1, 1, 1, 3], 8)
Done in 0.18230891227722168 .
start
([0, 0, 0, 0, 0, 0, 0, 0, 0, 0, 0, 0, 0, 0, 0, 0, 0, 0, 0, 0, 0, 0, 1, 0, 0, 0, 0, 1, 1, 0, 3], 6)
Done in 0.006891012191772461 .
end
([0, 0, 0, 0, 0, 0, 0, 0, 0, 0, 0, 0, 0, 0, 0, 0, 0, 0, 0, 0, 0, 0, 1, 0, 0, 0, 0, 1, 1, 0, 3], 6)
Done in 0.006652355194091797 .
index random
([0, 0, 0, 0, 0, 0, 0, 0, 0, 0, 0, 0, 0, 0, 0, 0, 0, 0, 0, 0, 0, 0, 1, 0, 0, 0, 0, 1, 1, 0, 3], 6)
Done in 0.007452487945556641 .
}

\begin{thebibliography}{99}
\addcontentsline{toc}{chapter}{Bibliografia}


\bibitem[Zen69]{heu} Zenon Zenon, \textit{Użyteczne heurystyki
    w~analizie}, Młody Technik, nr~11, 1969.

\end{thebibliography}

\end{document}


%%% Local Variables:
%%% mode: latex
%%% TeX-master: t
%%% coding: latin-2
%%% End:
